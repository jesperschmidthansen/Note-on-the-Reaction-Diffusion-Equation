
\section{The FTCS algorithm} \label{sect:ftcs}
%\section{Appendix B}
Before advicing a numerical scheme for the diffusion equation it is
illustrative to first see an example a scalar differential equation, and how this 
can be solved numerically. In particular, it will introduce some of the symbolism we will use 
later.

We consider the general first order problem 
\begin{equation}
  \label{eq:I:difscalar}
  \frac{\d c}{\d t} = c'(t) = f(c, t) \ , \ \ \ c(t_0)=c_0
\end{equation}
such that $c: I \rightarrow \R$, $I \subseteq \R$ and $f$ is known, of course. 
This defines an \emph{inital value problem}; for these problems we usually use $t$ 
as the free variable rather than $x$. 

We assume that the function $c$ can be represented by a  
Taylor series, i.e., by definition $c$ is analytical in all interior points in $I$. (We actually do not need this strict condition, but it will 
save us some headaches.) We then Taylor expand $c$ around an interior point $a \in I$
\begin{eqnarray}
	c(t) = c(a) + c'(a)(t-a) + \sum_{k=2}^\infty   \frac{c^{(k)}(a)}{k!}(t-a)^k \, . 
\end{eqnarray}
We will write this is in a slightly different way by introducing $\Delta t = t -a$
\begin{eqnarray}
  c(a + \Delta t) &=& c(a)+c'(a)\Delta t + \sum_{k=2}^\infty
  \frac{c^{(k)}(a)}{k!}\Delta t^k \nonumber \\ &=&
  c(a)+c'(a)\Delta t + \bigo(\Delta t^2)
\end{eqnarray}
where $\Delta t = t-a$. $\bigo$ is the remainder function (called
big-O) and the argument to $\bigo$ indicates the lowest order in
$\Delta t$, here $\Delta t^2$. Loosely speaking, we say that
$\bigo(\Delta t^{k})$ goes faster to zero than $\bigo(\Delta t^{k-1})$
as $\Delta t$ goes to zero. Re-arranging gives
\begin{equation}
  \label{eq:o1}
  c'(a) = \frac{c(a+\Delta t)-c(a)}{\Delta t} - \bigo(\Delta t)
\end{equation}
using $\bigo(\Delta t^2)/\Delta t = \bigo(\Delta t)$. 
Substitution into the original differential equation, Eq. (\ref{eq:I:difscalar}), and re-arranging 
gives  
\begin{equation}
  c(a+\Delta t) = c(a) + \Delta t f(c(a), a) + \bigo(\Delta t^2) \, .
\end{equation}
To use this in practise we can let $t_0=0$, that is, we have the
initial condition $c(0)=c_0$. Then, we have by substitution
\begin{eqnarray}
  c(\Delta t) &=& c_0 + \Delta t f(c_0, 0) +
  \bigo(\Delta t^2) \nonumber \\
  c(2\Delta t) &=& c(\Delta t) + \Delta t
	f(c(\Delta t), \Delta t) + \bigo(\Delta t^2) \nonumber \\
  &\ldots& \nonumber \\
  c((n+1)\Delta t) &  =& c(n\Delta t) +
  \Delta t f(c(n\Delta t), n\Delta t) + \bigo(\Delta t^2)
  \label{eq:I:iteration} \\
  &\ldots& \nonumber
\end{eqnarray}
This presents an iterative scheme: from $c(0)=c_0$ we evaluate the
function value at $t=\Delta t$, and from this the value at $t=2\Delta
t$, etc. This is the simplest numerical scheme we can compose, and it is called the
\emph{first order forward Euler scheme} or just the Euler scheme.

Usually we replace the argument with the iteration index such that
$c(n \Delta t) = c^n$. Eq. (\ref{eq:I:iteration}) is then
written as (leaving out the $\bigo$-function)
\begin{equation}
  c^{n+1} = c^n + \Delta t f(c^n, n\Delta t)
\end{equation}
The Euler scheme is called an explicit scheme because the
value at time $(n+1)\Delta t$ is given explicitly in terms of values
at time $n\Delta t$. It is important to note that explicit schemes are
so called non-symplectic and they cannot be used to solve problems
where there is a system constant, for example, a constant energy.

\bigskip

\noindent Moving on to the diffusion equation. Recall, for a Dirichelt problem we have 
\begin{equation}
  \frac{\partial c}{\partial t} = \frac{\partial^2 c}{\partial
    x^2}
\end{equation}
with boundary and initial values
\begin{equation}
	c(x,0) = f(x) \ \ \mbox{and} \ \ c(0,t) = c_0 \ \ c(L,
  t) = c_L \, .
\end{equation}
Again $c$ is analytical on its domain, $t \geq 0$, and $0\leq x \leq L$. 

Keeping $x$ fixed we can Taylor expand with respect to $t$ giving
\begin{equation}
  c(x, a+\Delta t) = c(x,a) + \left.\frac{\partial c}{\partial
    t}\right|_{t=a}\Delta t + \bigo(\Delta t^2) \, ,
\end{equation}
or equivalently
\begin{equation}
  \left. \frac{\partial c}{\partial t}\right|_{t=a} = 
  \frac{c(x, a+\Delta t)- c(x,a)}{\Delta t} - \bigo(\Delta t) \, .
\end{equation}
Now, keep $t$ fixed. We Tayler expand up to second order with respect to $x$, thus, 
around a point $0<b<L$ we get 
\begin{equation}
  c(b+\Delta x, t) = c(b,t) + \left.\frac{\partial c}{\partial x}\right|_{x=b}\Delta
  x + \left.\frac{1}{2}\frac{\partial^2 c}{\partial x^2}\right|_{x=b} \Delta x^2 +
  \bigo(\Delta x^3)
  \label{eq:I:2orderf}
\end{equation}
where $\Delta x = x-b$. There is an annoying first order derivative
here; thankfully we can eliminate this by noting that
\begin{equation}
  c(b-\Delta x, t) = c(b,t) - \left.\frac{\partial c}{\partial x}\right|_{x=b} \Delta
  x + \left.\frac{1}{2}\frac{\partial^2 c}{\partial x^2}\right|_{x=b} \Delta x^2 +
  \bigo(\Delta x^3)
   \label{eq:I:2orderb}
\end{equation}
Hence, addition of Eqs.(\ref{eq:I:2orderf}) and (\ref{eq:I:2orderb})
gives
\begin{equation}
  c(b+\Delta x, t) + c(b-\Delta x, t) = 2 c(b,t) + \left.\frac{\partial^2
    c}{\partial x^2}\right|_{x=b} \Delta x^2 + \bigo(\Delta x^4)
\end{equation}
or
\begin{equation}
  \left.\frac{\partial^2 c}{\partial x^2}\right|_{x=b} =
  \frac{c(b+\Delta x, t) - 2 c(b,t) + c(b-\Delta x, t) }{\Delta x^2} -
  \bigo(\Delta x^2) \label{eq:I:secondorder}
\end{equation}
Again, we replace the arguments with iteration index, i.e.,
\begin{eqnarray}
	c(x,t)&=&c_i^n \nonumber \, \\
	c(x,t+\Delta t)&=&c_i^{n+1} \, \nonumber \\ 
	c(b+\Delta x,t)&=&c_{i+1}^n \, \nonumber \\
	c(b-\Delta x,t) &=& c_{i-1}^n \,
	\end{eqnarray}
and we write the diffusion equation as (leaving out the $\bigo$s)
\begin{equation}
\frac{c_i^{n+1} - c_i^n}{\Delta t} \approx \frac{c_{i-1}^n -
  2c_i^n + c_{i+1}^n}{\Delta x^2} 
\end{equation}
readily giving the iterative scheme for point $i$
\begin{equation}
  c_i^{n+1} = c_i^n + \Delta t \left[\frac{c_{i-1}^n -2c_i^n +
    c_{i+1}^n}{\Delta x^2}\right] \, .
\end{equation}
This scheme is called the forward time centered space (FTCS) algorithm
and is also an explicit scheme. 

\begin{exerciseregion}
    \begin{exercise}    
    Consider the differential equation
    \begin{equation}
        \label{eq:expl1}
		c'(t) = 1 - c \ \text{with} \ c(0) = 0
    \end{equation}
	Solve this initial value problem. Then impliment the Euler scheme and solve the
	problem numerically. (Use your favorite programming language; if this
	is Java you are on your own!)  Test your implementation against the
    analytical result.
    \end{exercise}
    
    \noindent Note: The error we make at each time step can be evaluated
    directly from the higher order Taylor polynomial terms if the function
    $c$ is known. Use this fact if you wish.

 \begin{exercise}
  Let $c(x) = 1-x^2$, where $0 \leq x \leq 1$ and $c(0)=c(1)=0$. Use
  Eq. (\ref{eq:I:secondorder}) to calculate the second order
  derivative. Compare to the analytical result. Make a plot of the
  error as function of $\Delta x$.   
\end{exercise}

\begin{exercise}
  Implement the FTCS alorithm. Use this to solve a diffusion problem
  where the analytical solution is known. Discuss possible errors.
\end{exercise}

\end{exerciseregion}
