 It will take some extra effort, but
it will be worth it! First, define the vector space over the real numbers,
having the following set of functions defined on the interval $I=[-L; L]$
\begin{eqnarray}
	V = \left \{ 1, \cos\left(\frac{\pi}{L}x\right), 
	\cos\left(\frac{2\pi}{L} x\right), \ldots, \sin\left(\frac{\pi}{L}x\right), 
	\sin\left(\frac{2\pi}{L}x \right), \ldots \right \} \, ,
\end{eqnarray}
as well as the usual addition and multiplication operations. 
Note, the vectors in $V$ are real valued functions defined on $I$ and not tuples of 
scalars as taught in elementary linear algebra. Clearly, the axioms for a vector space is fulfilled. 
For example, if $n\in \mathbb{N}$ and $m \in
\mathbb{N}_+$ we have $\cos(n \pi x/L) \in V$ and $\sin(m \pi x/L) \in V$, so if
$K \in  \mathbb{R}$, then
\begin{eqnarray}
	K\left(\cos\left(\frac{n\pi x}{L}\right) + 
	\sin\left(\frac{n \pi x}{L}\right)\right)  = 
	K\cos\left(\frac{n\pi x}{L}\right) + K\sin\left(\frac{n \pi x}{L}\right)\, , 
\end{eqnarray}
which is the axiom for distribution of scalar multiplication. 

The vectors in $V$ are linearly independent meaning that if the linear
combination is zero,
\begin{eqnarray}
	\label{eq:linvector}
\frac{a_0}{2} + \sum_{n=1}^\infty a_n \cos\left(k_n \, x\right) 
	+ \sum_{n=1}^\infty b_n \sin\left(k_n\, x\right) = 0 \, ,
\end{eqnarray}
then $a_0, a_n$, and $b_n$ are all zero. This in turn means that the elements in $V$ form a
vector basis. Notice, that the lhs of Eq. (\ref{eq:linvector}) is the Fourier series. 

We can equip the vector space with an \emph{inner product}. This provides 
some addition structure which we can exploit. Since our vectors are real
valued function on $I$, we explicitly define the inner product here.  
Let $f,g \in V: I \rightarrow \R$, where $I = [a; b]$, then the inner product 
on $I$ is defined as
\begin{eqnarray}
	\langle f,g\rangle= \int_a^b f(x) g(x) \, \d x \, .
\end{eqnarray}
We say that $f$ and $g$ are orthogonal on $I$ if $\langle f,g \rangle = 0$. From
the inner product we can define the vector norm as 
\begin{eqnarray}
	|| f || = \sqrt{\langle f, f\rangle} \, .
\end{eqnarray}

\begin{theorem}
	All vectors in $V$ have norm one.
\end{theorem}	

\begin{proof}
	(i) $\langle 1, 1\rangle = 
\end{proof}	



\begin{theorem}
	The vectors in $V$ span an orthonormal basis on $I=[-L;L]$. 
\end{theorem}

\begin{proof}
	We know that the vectors form a basis. We then need to show that 
	the inner product between each pair is zero. We can do this for 
	a few cases and leave the others as exercise.

	(i) Let $f=1$ and $g=\cos(n\pi x/L)$. We then have
	\begin{eqnarray}
		\left\langle 1, \cos\left(\frac{n \pi x}{L}\right) \right\rangle = \int_{-L}^L
		\cos\left( \frac{n\pi x}{L} \right) = 0 \ \forall n \in \mathbb{N}
	\end{eqnarray}
	We now quickly see that 	
	\begin{eqnarray}
	\left\langle 1, \sin\left(\frac{n \pi x}{L}\right) \right\rangle = 0 \ \forall n \in \mathbb{N}
	\end{eqnarray}
	(ii) Let $f=\cos(n \pi x/L) \cos(m \pi x /L)$ where $n,m \in \mathbb{N}$

\end{proof}	





