
If the system is non-reactive, $r=0$, the RD-equation reduces to the diffusion equation. If the 
diffusion coefficient is constant this is in one dimension (and Cartesian
coordinates, of course)
\begin{equation}
	\frac{\partial c}{\partial t} = D \frac{\partial^2 c}{\partial x^2} \, .
\end{equation}
To solve this problem we need an IC, $c(x,0)$, and BCs which can be of the Dirichlet type, Neuman type
or a mix of the two. 

Before we explore the diffusion equation, we must first have some fundamental knowledge about   
the famous Fourier series and the Euler-Fourier equations as they play a key role in 
this exploration. 

\section{\label{sect:fourier}Periodic functions and the Fourier series}

\begin{wrapfigure}{R}{0.4\textwidth}
	\centering
	\includegraphics[width=0.35\textwidth]{figs/sawtooth.eps}
	%\caption{\label{fig:img2} Perturbation}
\end{wrapfigure}
\paragraph{}
\vspace*{-\parskip}

In this text, a periodic function is a function defined on a subset $S$ in $\mathbb{R}$
and which has repeating values such that for $p \neq 0$ 
\begin{equation}
	f(x)=f(x+p)  \, 
\end{equation}
where $x \in S$ and $x + p \in S$. $p$ is called the function period. 
$p$ is not unique; the least value for $p$ is referred to as 
the \emph{fundamental period}, e.g., the functions cosine and sine 
have fundamental period 2$\pi$. Another, and less trivial example, is the \emph{sawtooth
function}
\begin{equation}
 s(x) = x - \lfloor x \rfloor \, ,
\end{equation}
here  $\lfloor x \rfloor$ is the floor function, defined such that $\lfloor -0.5 \rfloor = -1,
\lfloor 0.5 \rfloor = 0$, $\lfloor 1 \rfloor = 1$, and so forth. 
The sawtooth function as a fundamental period of 1, and since the left and right function
limits in points $x=\ldots -2,-1,0,1,2, \ldots$ exist, but are different $s$ features
\emph{jump discontinuities} in these points. 

Another way of defining the sawtooth function is to \emph{periodically expand} the 
function $f(x) = x$, where $0 \leq x < 1$. The expansion is then given as 
\begin{eqnarray}
	s(x) = f(x + np) \ , 
\end{eqnarray}
with $p=1$ and $n \in \mathbb{Z}$.

The sawtooth function is defined on the entire real line and $S=\R$. We can define another (and related) 
periodic function with fundamental period 1 using the periodic expansion of $f(x)=x$, 
but where $0 < x < 1$. Here $S=\R \backslash \{\mathbb{Z}\}$. 

Now, a question relevant to us is how a periodic function can be "represented" by a function 
series. What we mean by a representation will hopefully be clear as we go on. 
The well known Taylor series (or in general any power series) will not be a good choice 
since a necessary condition here is that the function must be  
infinitely differentiable on its domain. $s$ above clearly does not fulfill 
this when $x$ takes integer values, $\ldots -2,-1,0,1,2, \ldots$. 
Even if the function is analytical and therefore has a Taylor series, 
this is not very practical choice since the Taylor 
series converges slowly for periodic functions. Then, we suggest to represent the function as 
a series of known periodic functions, that is, 
\begin{equation}
	\label{eq:Fouriereq}
	F(x) = \frac{a_0}{2} + \sum_{n=1}^\infty a_n \cos\left(\frac{n\pi }{L} \, x\right) 
	+ \sum_{n=1}^\infty b_n \sin\left(\frac{n\pi }{L} \, x\right) \, .
\end{equation} 
This is the famous \emph{Fourier series}, and the 
coefficients $a_0, a_1, a_2, \ldots$, $b_1, b_2, b_3, \ldots$ are called the 
\emph{Fourier coefficients}, and $n$ the \emph{wave number}. 
By the way, an important point here is that a sum of periodic functions is also periodic. 

\begin{example}
	The function $f(x) = 2\cos(2\pi x/L) + 3\sin(4\pi x/L)$ has two non-zero 
	Fourier coefficients, namely, $a_2=2$ and $b_4=3$. 
\end{example}

\begin{example}
	The general solution in Eq. (\ref{eq:2ndorderseries}) is a Fourier series with $a_n = 0$,
	$\forall n \in \mathbb{N}_+$.
\end{example}

\noindent $L$ that enters in the denominator needs to be clarified. $n=1$ is the 
\emph{fundamental term} or \emph{fundamental mode} and note that 
$\cos(\pi x/L)$ and $\sin(\pi x/L)$ have one period in the interval $[-L; L]$, 
that is, the period for the fundamental mode is $p=2L$. 
For $n=2$ the period is $L$, $n=3$ the period is $2/3L$, and so forth;  
in general the relation between the mode period and $L$ is 
\begin{eqnarray}
	p = \frac{2L}{n} \, .
\end{eqnarray}
Often the Fourier series is expressed through the \emph{wave vector} $k_n=2\pi/p = n\pi/L$, that is,
\begin{equation}
	F(x) = \frac{a_0}{2} + \sum_{n=1}^\infty a_n \cos\left(k_n \, x\right) 
	+ \sum_{n=1}^\infty b_n \sin\left(k_n\, x\right) \, .
\end{equation} 
The mathematician will rightfully ask: "Exactly when can a periodic function be represented 
(whatever that means) by a Fourier series?". The more insightful answer to this question 
requires a rather long introduction to different types of series 
convergence which we will avoid. For our purpose, it suffice to state that a
function can be represented by a Fourier series if the function is
\begin{enumerate}
	\item piecewise continuous on the interval $x_0 < x < x_0 + L$, 
		that is,  the function domain can be partitioned into a finite set of open 
		intervals where the function is continuous, and 
	\item bounded on the domain, that is, there exists an 
		$M \in \R$ such that $|f(x)| \leq M$, for all $x$ in the domain $S$.
\end{enumerate}
The sawtooth function is clearly bounded. Moreover, we can partition each of the intervals 
$\ldots, -1 \leq x < 0, 0 \leq x < 1, \ldots$ into one open continuous interval 
$\ldots, -1 < x <0$, $0 < x < 1, \ldots$ Therefore, $s$ can be represented by a
Fourier series. That the sawtooth function can be represent by $F$ does not mean that the two
functions are equal in every point in the domain: $s(x) \neq F(x)$ 
in the jump discontinuities even if the two conditions above are fulfilled. We
therefore write the representation as
\begin{equation}
	\label{eq:sawtoothFourier}
	s(x)  \sim F(x) \, .
\end{equation}

Now, it remains to find the Fourier coefficients in the representation. 
To this end we need to know about \emph{orthogonality} of functions. Let $f,g: I \rightarrow \R$, 
where $I = [a; b]$, then the \emph{inner product} on $I$ is defined as
\begin{eqnarray}
	\label{eq:innerdef}
	\langle f,g\rangle= \int_a^b f(x) g(x) \, \d x \, .
\end{eqnarray}
We say that $f$ and $g$ are orthogonal on $I$ if $\langle f,g \rangle = 0$. 

We shall benefit from the following properties of the sine and cosine functions on
$I=[-L;L]$ 
\begin{enumerate}
	\item $\cos\left(\frac{n\pi x}{L}\right)$ and  
		$\sin\left(\frac{m\pi x}{L}\right)$ are orthogonal for all $n,m
		\in \mathbb{N}_ +$
	\item  $\cos\left(\frac{n\pi x}{L}\right)$ and  
		$\cos\left(\frac{m\pi x}{L}\right)$ are orthogonal if $n \neq
		m$, otherwise if $n=m$ the inner product is $L$.
	\item  $\sin\left(\frac{n\pi x}{L}\right)$ and  $\sin\left(\frac{m\pi
		x}{L}\right)$ are orthogonal if $n \neq
		m$, otherwise if $n=m$ the inner product is $L$.
\end{enumerate}
These statements can be verified from direct calculations. For example, 
\begin{eqnarray}
	\left \langle \cos\left(\frac{n\pi }{L}x\right), \sin\left(\frac{m\pi
	}{L}x\right) \right \rangle &=&  \int_{-L}^L \cos\left(\frac{n\pi}{L}x\right) \sin\left(\frac{m\pi
	}{L}x\right) \, \d x
	\nonumber \\
	&=& \frac{1}{2} \int_{-L}^L
	\sin\left(\frac{(n+m)\pi}{L}x\right)+\sin\left(\frac{(m-n)\pi}{L}x\right) \, \d x 
	\nonumber \\
\end{eqnarray}
using the product rule $2\cos(a)\sin(b) = \sin(a+b) + \sin(a-b)$. This integral is
always zero, hence, the functions are orthogonal. 

We are now ready for the section's main theorem
\begin{theorem}
	If the Fourier series representation of $f:\mathbb{R} \rightarrow \mathbb{R}$ 
	equals the function, i.e., if $f(x)=F(x)$ for all $x \in \R$, then
	\begin{description}[style=unboxed,leftmargin=0.5cm]
		\item[1] $\langle f(x), 1 \rangle = La_a$
		\item[2] $\left \langle f(x), \cos\left(\frac{n \pi}{L}x\right)
			\right\rangle = L a_n$
		\item[3] $\left \langle f(x), \sin\left(\frac{n \pi}{L}x\right)
			\right\rangle = L b_n$
	\end{description}	
	on $I=[-L; L]$, and where $n\in\mathbb{N}_+$. These are the \emph{Euler-Fourier
	equations.}  
\end{theorem}

\begin{proof}
	\begin{enumerate}
		\item Since $f(x)=F(x)$  
			\begin{eqnarray}
				\label{eq:f=F}
				f(x) =  \frac{a_0}{2} + \sum_{n=1}^\infty a_n \cos\left(k_n \, x\right) 
										+ \sum_{n=1}^\infty b_n \sin\left(k_n\, x\right) \, .
			\end{eqnarray}
			Integrating both sides from $-L$ to $L$ 
		\begin{eqnarray}
			 \int_{-L}^{L}f(x) \d x = \frac{a_0}{2} \int_{-L}^L \d x + \int_{-L}^L \sum_{n=1}^\infty a_n 
				\cos\left(\frac{n\pi }{L} \, x\right) \d x + \int_{-L}^L \sum_{n=1}^\infty b_n \sin\left(\frac{n\pi 
				}{L} \, x\right) \d x \nonumber \\
		\end{eqnarray}
		We can interchange integration and summation in the last two terms; this is due to the fact that 
		the two function series converges uniformly. Then, for all $n \in \mathbb{N}_+$ we have  
		\begin{equation}
			\int_{-L}^L  \cos\left(\frac{n\pi }{L} \, x\right) \d x = 0 \ \text{and} \ 
			\int_{-L}^L  \sin\left(\frac{n\pi }{L} \, x\right) \d x = 0 \ , 
		\end{equation}
		and therefore
		\begin{equation}
			\label{eq:a0}
			 \int_{-L}^L f(x) \d x = \frac{a_0}{2}   \int_{-L}^L \d x
			 \Rightarrow a_0 = \frac{1}{L} \int_{-L}^L f(x) \d x \, . 
		\end{equation}
		This is the result we are after.
		
	\item  Multiplying Eq. (\ref{eq:f=F}) with $\cos(m \pi x/L)$, $m=1,2,3, \ldots$ 
		and integrate from $-L$ to $L$
		\begin{eqnarray}
			\int_{-L}^{L}\cos\left(\frac{m\pi }{L} \, x\right) f(x)\d x &=&
				 \frac{a_0}{2} \int_{-L}^L \cos\left(\frac{m\pi }{L} \, x\right) \d x 
				 \nonumber \\
				 &+& \int_{-L}^L \sum_{n=1}^\infty a_n\cos\left(\frac{m\pi }{L} \, x\right) 
				 \cos\left(\frac{\pi n}{L} \, x\right) \d x 
				\nonumber \\
				&+& \int_{-L}^L \sum_{n=1}^\infty b_n \cos\left(\frac{m\pi }{L} \, x\right)   \sin\left(\frac{n\pi }{L} 
				\, x\right) \d x 
				\nonumber \\
			\label{eq:pointii}
		\end{eqnarray}
		The first term on the right-hand side is zero. For the second term 
		we again interchange the integration and summation, giving raise to
		integrals with relations
		\begin{eqnarray}
			\int_{-L}^L \cos\left(\frac{m\pi }{L} \, x\right) \cos\left(\frac{\pi n}{L} \, 
			x\right) \d x = L \delta_{mn} .
		\end{eqnarray}
		comming the orthogonality properties above. $\delta_{mn}$ is the Kronecker delta, 
		$\delta_{mn}=1$ if $m=n$, otherwise if $m \neq n$ then $\delta_{mn}
		= 0$. This means that all terms are zero except for $m=n$. The third term 
		yields integrals on the form
		\begin{eqnarray}
			\int_{-L}^L \cos\left(\frac{m\pi }{L} \, x\right)
			\sin\left(\frac{n\pi }{L}	\, x\right) \, \d x = 0 
		\end{eqnarray}
		since cosine and sine are ortohgonal on $I$. 

		Therefore, Eq. (\ref{eq:pointii}) simplifies to
		\begin{equation}
			\int_{-L}^{L}\cos\left(\frac{m\pi }{L} \, x\right) f(x)\d x  = a_n \int_{-L}^L  
			\cos\left(\frac{\pi n}{L} \, x\right) \cos\left(\frac{\pi n}{L} \, x\right) \d x = L a_n \ , 
		\end{equation}
		that is,
		\begin{equation}
			\label{eq:an}
			a_n = \frac{1}{L} \int_{-L}^L \cos\left(\frac{n\pi }{L} \, x\right)
			f(x)\d x \, . 				\ \ (n=1,2,3, \ldots)
		\end{equation}
		Notice that if we allow $n=0$ we arrive at Eq. (\ref{eq:a0}). This explains the choice 
		of symbol for the constant term in the Fourier series.
	\end{enumerate}
	The last equation is left as an exercise.
\end{proof}


\noindent We now make a very important definition: The Fourier representation of a function $f$ is the Fourier
series with coefficients given by the Euler-Fourier equations if these exist.  

\begin{example}
	Find the Fourier coefficients of the sawtooth function. Then write-down the Fourier series representation.
	
	\bigskip
	
	\noindent $a_0$: We use $L=1$, then from Eq. (\ref{eq:a0}).
	\begin{equation}
		a_0 = \int_{-1}^1 s(x) \, \d x = \int_{-1}^0 x + 1\, \d x + \int_{0}^1 x \, \d x = 1 \ .  
	\end{equation}
	

	\noindent $a_n$: Rather than simply embarking on the tedious task of integrating Eq. 
	(\ref{eq:an}) we can re-arrange the integral, which leaves the problem straight forward. 
	Substituting for $s$ we get 
	\begin{eqnarray}
		\int_{-1}^1 s(x) \cos(n\pi x) \, dx &=& \int_{-1}^0 (x+1) \cos(n\pi x) \, \d x + \int_{0}^1 
			x\cos(n\pi x) \, \d x \nonumber \\
			&=& \int_{-1}^1 x \cos(n\pi x) \d x + \int_{-1}^0 \cos(n \pi x) \, \d x
	\end{eqnarray}
	The first term is an integral over an odd function and this gives zero. For the last 
	term we get 
	\begin{equation}
		 \int_{-1}^0 \cos(n \pi x) \, \d x = \left. \frac{1}{n\pi} \sin (n\pi x)\right|_{-1}^0 = 0 \, .
	\end{equation}
	thus, $a_n=0, n = 1,2,3,\ldots$ 

	\noindent $b_n$: Following the protocol above we have  
		\begin{eqnarray}
		\int_{-1}^1 s(x) \sin(n\pi x) \, dx  
			= \int_{-1}^1 x \sin(n\pi x) \d x + \int_{-1}^0 \sin(n \pi x) \, \d x
			\label{eq:sfourierb2}
		\end{eqnarray}
		Using integration by parts the first integral is evaluated to 
		\begin{eqnarray}
			 \int_{-1}^1 x \sin(n\pi x) \d x = \left .  - \frac{1}{n\pi}
			 x\cos(n\pi x)\right|_{-1}^1 - \int_{-1}^1 \sin(n\pi x) \, \d x
		 \end{eqnarray}
	
	\begin{wrapfigure}[17]{R}{0.4\textwidth}
	\centering
	\includegraphics[width=0.35\textwidth]{figs/fouriersawtooth.eps}
	\caption*{}
	\end{wrapfigure}
%\paragraph{}
%\vspace*{-\parskip}



	Sine is an odd function and the second integral is zero. Using the two relations
		\begin{equation}
			\cos(n\pi) = (-1)^n \ \text{and} \ \cos(n\pi) = \cos(-n \pi)
		\end{equation}
		we then get $\int_{-1}^1 x \sin (n\pi x) = -2(-1)^n/n\pi$. The second integral in Eq. 
		(\ref{eq:sfourierb2}) evaluates to 
		\begin{equation}
			\int_{-1}^0 \sin(n \pi x) \, \d x = - \frac{1 - (-1)^n}{n\pi}
		\end{equation}
		and we get 
		\begin{equation}
			b_n = -\frac{1 + (-1)^n}{n \pi} \, .
		\end{equation}	 
		The Fourier series representation of the sawtooth function is therefore
		\begin{equation}
			\label{eq:fouriersaw}
			s(x) \sim \frac{1}{2} - \sum_{n=1}^\infty  \frac{1 + (-1)^n}{n \pi} \sin(n\pi x) 
		\end{equation}
		Note that when $n$ is odd the nominator in the summation is zero, and we can simply 
		let $n$ run over the even natural numbers.
\end{example}


\noindent From Eq. (\ref{eq:fouriersaw}) we see that for $x \in \Z$ the Fourier representation yields 
the value 1/2, but this is not the value of the sawtooth function itself. 
It is a general feature of the Fourier representation that at point $x_0$ it has value of the average of the right and 
left limits of the function $f$ it represents (so these must exist). That is, we have
\begin{eqnarray}
	F(x_0) = \frac{ \lim_{x \rightarrow x_0^+}f(x) + \lim_{x \rightarrow
	x_0^-}f(x) }{2} \, , \ \  x_0 \in \mathbb{R} \, . \nonumber \\
\end{eqnarray}
In points where the function is continuous 
the Fourier representation value then equals the function value. 

When visualizing the Fourier series we cannot plot the actual series as this means adding 
infinitely many terms. Rather we plot the \emph{partial Fourier sums} (or just 
the partial sums); for the sawtooth function we can write the partial sum from 1 to $m$ as 
\begin{eqnarray}
	s_m(x) = \frac{1}{2} - \sum_{n=1}^m  \frac{1 + (-1)^n}{n \pi} \sin(n\pi x) \, .
\end{eqnarray}
We can now define the remainder $R_m(x) = s(x) - s_m(x)$. The remainder $R_{50}$ is plotted in Fig. 
\ref{fig:remainder-fourier} (a) for the sawtooth function. 
\begin{figure}[h]
	\begin{center}
		\includegraphics[scale=0.35]{figs/remainderFourier.eps}
		\caption{\label{fig:remainder-fourier}
			(a) The remainder function for the sawtooth function.
			(b) The Fouier coefficient, $b_n$, as function of $n$.
		}
	\end{center}
\end{figure}
We see a large disagreement in neighbourhoods around 
the points of discontinuities, which is manifested as large amplitude oscillations. This is known 
as the \emph{Gibbs phenomenon}. 

In Fig. \ref{fig:remainder-fourier} (b) the Fourier coefficients, $b_n$, are plotted as function of $n$ (for 
$n$ odd $b_n$ is not shown). We see that the coefficients decrease, giving us the hope that the Fourier series 
actually converges. Needless to say, a convincing argument requires a proof, which we do not pursue here. 

\begin{exerciseregion}
	\begin{exercise}
		\begin{enumerate}
			\item Proof that Eq. (\ref{eq:innerdef}) defines an inner product. (You may have 
				to recall the exact definition of an inner product.) 
			\item Show that $f(x) = x$ and $g(x)=x^2 - 1$ are orthogonal on $I=[-1,1]$, but not on $I=[-1;2]$. 
		\end{enumerate}		
	\end{exercise}

	\begin{exercise}
		Proof the Euler-Fourier equation $ \langle f(x),
		\sin\left(\frac{n\pi}{L}x\right)\rangle = Lb_n$
	\end{exercise}
		
	\begin{exercise}
		Consider the function
		\begin{eqnarray}
			f(x) = 
			\begin{cases} 
			      0 & -L \leq x < 0 \\
				  c_0 & 0 \leq x < L
   			\end{cases}
		\end{eqnarray}
		\begin{enumerate}
			\item Sketch a periodic expansion of $f$.  Where do you expect the
				Gibbs phenomenon to arise?
			\item Find the Fourier series representation of $f$.
			\item Plot the partial sums and the remainder for $m=3,9,99$. (Why the odd index?)
		\end{enumerate}			
	\end{exercise}

	\begin{exercise} \label{ex:fourierevenodd}
		(Mandatory) An even function is defined as a function where $f(x) = f(-x)$ and an odd function a function
		where $f(x) = -f(-x)$. For example, cosine is even and sine odd. It is quite easy to show 
		that, for example,  
		\begin{itemize}
			\item the sum of even functions results in an even function,
			\item a sum of odd functions results in an odd function,
			\item if $f$ is even \[ \int_{-L}^L f(x) \d x = 2\int_0^L f(x) \d x\] 
			\item if $f$ is odd \[ \int_{-L}^L f(x) \d x = 0 \] 
		\end{itemize}	
		Therefore, if $f$ is even the Fourier series is also 
		even resulting in the \emph{cosine series}
		\begin{eqnarray}
			\frac{a_0}{2} + \sum_{n=1}^\infty a_n \cos\left(\frac{n\pi}{L}x\right) \, .
		\end{eqnarray}
		If $f$ is odd the Fouier series is odd giving the \emph{sine series}
		\begin{eqnarray}
			\sum_{n=1}^\infty b_n \sin\left(\frac{n\pi}{L}x\right) \, .
		\end{eqnarray}
		Consider the function $f(x) = x^2$, $-2<x<2$, i.e., $L=2$.
		Find the Fourier series representation. Is the result consistent with the exercise preamble?
	\end{exercise}
	
	\begin{exercise}
		Prove that the product of two odd functions is an even function. Is the product between to 
		even functions then an odd function?
		\end{exercise}

	\begin{exploration}
		In many "real life" situations you may not have a function definition for the periodic function or the 
		Euler-Fourier integrals are untractable. Then the coefficients 
		can be found by numerical methods. Go to Appendix \ref{appendix:fourier}.
	\end{exploration}
\end{exerciseregion}

\section{The diffusion equation with Dirichlet BCs}
After Sect. \ref{sect:fourier} we are ready to explore the diffusion equation
\begin{eqnarray}
	\frac{\partial c}{\partial t} = D\frac{\partial^2c}{\partial x^2} \, .
\end{eqnarray}
We will use the general IC 
\begin{eqnarray}
	c(x, 0) = f(x) \, ,
\end{eqnarray}
and Dirichlet BCs 
\begin{eqnarray}
	\label{eq:bcstart}
	c(0, t) = c(L,t) = 0 \, .
\end{eqnarray}
The starting point is that we assume $c$ can be written as a product of two function, $X$ and $T$, which 
depends on $x$ and $t$, respectively. That is,
\begin{eqnarray}
	c(x,t) = X(x)T(t) \, .
\end{eqnarray}
This is referred to as \emph{separation of variables}. Of course, factorizing a function in this manner
is not generally true.
\begin{question}
Can you give examples of functions that are not separable and some that are separable?
\end{question}

\noindent Inserting into the diffusion equation we obtain
\begin{eqnarray}
	\frac{\partial}{\partial t} X(x)T(t) = D \frac{\partial^2}{\partial x^2} X(x) T(t) \ \Rightarrow 
	X(x) \frac{\d}{\d t} T(t) = D T(t) \frac{\d^2}{\d x^2} X(x)  \, .
\end{eqnarray}
If $T \neq 0$ for all $t>0$ and $X \neq 0$ for interior points $0 < x < L$, we can rearrange this giving
\begin{eqnarray}
	\frac{1}{D T} \frac{\d T}{\d t} = \frac{1}{X}\frac{\d^2X}{\d x^2} \, .
\end{eqnarray}
We have dropped the explicit dependencies on $x$ and $t$. 

Now, as $t$ (the time) increases the function $T$ will in general vary - how we do not know at the moment. 
However, the function $X$ does not vary as $t$ varies, so the right-hand side is a constant 
with respect to $t$. Likewise, as $x$ varies the left-hand side is constant since $T$ is independent of $x$. 
The constant of motion (with respect to $t$ and $x$) must be the same, and we have
\begin{eqnarray}
	\frac{1}{D T} \frac{\d T}{\d t} = -\alpha  \ \text{and} \ \frac{1}{X}\frac{\d^2X}{\d x^2} = -\alpha\, .
\end{eqnarray}
That we chose the constant to be $-\alpha$ is a matter of convenience. If
$\alpha = 0$ we get the trivial time independent solution $X(x) = 0$ due to the
boundary conditions which we avoid. 

Before we continue, it is worth to highlight what just happened: Assuming that $c$ can be factorized into 
a product of $X$ and $T$ we have transformed our (linear) 
partial differential equation into a problem of two (linear) ordinary differential 
equation for $X$ and $T$; and we can solve these two equations. 

We start out by solving for $X$. Rearranging once again, we get
\begin{eqnarray}
	\frac{\d^2 X}{\d x^2} + \alpha X = 0 \, .
\end{eqnarray}
We need the boundary conditions to continue. From Eq. (\ref{eq:bcstart}) 
\begin{eqnarray}
	c(0, t) = X(0)T(t) = 0 \ \Rightarrow X(0) = 0\, ,
\end{eqnarray}
because $T \neq 0$.  The same is true for $x=L$, and we have Dirichlet BCs $X(0)=X(L)=0$. 
We solved this exact boundary value problem in Sect. \ref{stst:sectBVP}, where
we could safely argue that $\alpha > 0$ since both the rate constant and the
diffusion coefficient are positive. We cannot use this argument here, but since 
relation 
\begin{eqnarray}
	\sqrt{\alpha} = \frac{n\pi}{L} \, , 
\end{eqnarray}
is still valid we see that $\alpha$ must be real and positive. Then
\begin{eqnarray}
	\alpha = \left( \frac{n \pi}{L} \right)^2 \ \text{with} \ n \in \mathbb{N}_+ \, . 
\end{eqnarray}
So for any $n \in \mathbb{N}_+$ we have the solution
\begin{eqnarray}
	X_n(x) = \chi_n \sin\left(\frac{n\pi}{L} x\right) \, ,
\end{eqnarray}
where $\chi_n$ is a constant (dependent on $n$).

$\alpha$ depends on $n$, and therefore the solution to $T$ also depends on
$n$. This for any $n \in \mathbb{N}_+$ we have the problem
\begin{eqnarray}
	\frac{\d T_n}{\d t} = -\alpha D T_n \, ,
\end{eqnarray}
with the general solution
\begin{eqnarray}
	T_n(t) = \tau_n e^{-(n\pi/L)^2Dt} \, .
\end{eqnarray}
We will not use the IC just yet. 

From above we see that there are infinite many solutions to $c$, namely, 
\begin{eqnarray}
	c_n(x,t) = X_n(x)T_n(t) = b_n  e^{-(n\pi/L)^2Dt}  \sin\left(\frac{n\pi}{L} x\right) \, ,
\end{eqnarray}
where we have set $b_n = \chi_n\tau_n$. Invoking the superposition principle this means that the 
general solution is
\begin{eqnarray}
	\label{eq:soldifdibc}
	c(x,t) = \sum_{n=1}^\infty c_n = \sum_{n=1}^\infty b_n  e^{-(n\pi/L)^2Dt}  \sin\left(\frac{n\pi}{L} x\right)
	\, .
\end{eqnarray}
We still need a particular solution. We now use the IC, $c(x,0)=f(x)$, that is, from the general 
solution given in Eq.(\ref{eq:soldifdibc}) we have at $t=0$
\begin{eqnarray}
	\label{eq:soldiffDirichelt}
	c(x, 0) = f(x) = \sum_{n=1}^\infty b_n  \sin\left(\frac{n\pi}{L} x\right) \, .
\end{eqnarray}
The IC is given by a Fourier sine series (and is odd). Thus, we can obtain the coefficients $b_n$ from the 
Euler-Fourier equation
\begin{eqnarray}
	\label{eq:bnf}
	b_n = \frac{1}{L}\left\langle f(x), \sin\left(\frac{n\pi}{L} x\right) \right \rangle= 
	 \frac{2}{L} \int_{0}^L f(x) \sin\left(\frac{n\pi}{L} x\right) \, \d x \ ,
\end{eqnarray}
since the product of two odd functions is even, see exercise \ref{ex:fourierevenodd}. 

\begin{question}
	Can you show that Eq. (\ref{eq:soldifdibc}) is actually a solution to the diffusion equation with zero Dirichlet BCs?
\end{question}	

\begin{example}
	The solution to the diffusion equation with Dirichlet BCs and IC
	\[
		c(x,0) = c_0 
	\] 
	is given by a sine series with Fourier coefficients
	\begin{eqnarray}
		b_n &=& \frac{2c_0}{L} \int_0^L \sin\left( \frac{\pi n}{L} x \right) \nonumber \\
			&=& -\frac{2c_0}{n\pi}( (-1)^n - 1) \, .
	\end{eqnarray}
	Notice that $b_n = 0$ for $n$ even and $b_n = 4c_0/n\pi$ for $n$ odd.
\end{example}	

\begin{wrapfigure}{R}{0.4\textwidth}
	\centering
	\includegraphics[width=0.35\textwidth]{figs/diffDBC.eps}
	%\caption{\label{fig:img2} Perturbation}
\end{wrapfigure}


\noindent The result in Eq. (\ref{eq:soldifdibc}) is a central result and is used in the more 
general case where the Dirichlet BCs read
\begin{eqnarray}
	\label{eq:diffc0stst}
	c(0,t) = c_1 \ \text{and} \ c(L,t)=c_2 \, .
\end{eqnarray}
First, we expect that for $t \rightarrow \infty$ the system reaches the steady state,
which we denote $c_\text{st}$; this follows by definition
\begin{eqnarray}
	\frac{\d^2 c_{\text{st}}}{\d x^2} = 0 \, ,
\end{eqnarray}
with BCs in Eq. (\ref{eq:diffc0stst}). Using the method from Sect. \ref{sect:BV} we can 
easily solve this giving 
\begin{eqnarray}
	c_\text{st}(x) = ax + c_1 \, 
\end{eqnarray}
where $a = (c_2-c_1)/L$. We now define the function
\begin{eqnarray}
	\overline{c} (x,t) = c(x,t) - c_\text{st}(x) \, , 
\end{eqnarray}
where we substract the steady-state case from $c$. Note that 
\begin{eqnarray}
	\frac{\partial c}{\partial t} = \frac{\partial \overline{c}}{\partial t} 
\end{eqnarray}
and
\begin{eqnarray}
	\frac{\partial^2 c}{\partial x^2} = \frac{\partial^2 \overline{c}}{\partial x^2} 
	- \frac{\d^2  c_{\text{sf}}}{\d x^2} = \frac{\partial^2 \overline{c}}{\partial x^2}
	\, .
\end{eqnarray}
This means that $\overline{c}$ also fulfilles the diffusion equation
\begin{eqnarray}
	\frac{\partial \overline{c}}{\partial t} = D\frac{\partial^2\overline{c}}{\partial{x}^2} \, .
\end{eqnarray}
Moreover, the new function $\overline{c}$ has BCs
\begin{eqnarray}
	\overline{c} (0,t) &=& c(0,t) - c_\text{st}(0) = 0 \, ,\\
	\overline{c} (L,t) &=& c(L,t) - c_\text{sf}(L) = 0 \, ,
\end{eqnarray}
and IC
\begin{eqnarray}
	\overline{c}(x,0) = c(x,0) - c_\text{st}(x) = f(x) \, .
\end{eqnarray}
This is just the problem we solved above! Therefore, the solution to $\overline{c}$ is 
just the Fourier sine series and, in turn, this means
\begin{eqnarray}
	c(x,t) = c_\text{st}(x) + \overline{c}(x,t)=ax+c_1 + \sum_{n=1}^\infty b_n \sin\left(\frac{n\pi}{L}x\right) \, ,
\end{eqnarray}
where $b_n$ is found directly from Eq. (\ref{eq:bnf}) 
\begin{eqnarray}
	b_n = \frac{2}{L} \int_0^L \left[ c(x,0) - c_\text{st}(x) \right] \sin\left( \frac{\pi n}{L} x \right) \, \d x \, .
\end{eqnarray}

\begin{exerciseregion}

	\begin{exercise}
		Solve the diffusion equation on the domain $0 \leq x \leq 1$ with Dirichlet BCs 
		\begin{eqnarray}
			c(0,t)=c(1,t)=0 \, ,
		\end{eqnarray}
		and IC $c(x,0)=x$. In your favorite plotting program, 
		plot the concentration profiles $c(x,t_0), c(x,t_1), \ldots$ for a few different 
		times. 
	\end{exercise}	

	\begin{exercise}
		Solve the diffusion equation on the domian $0 \leq x \leq 1$ with Dirichlet BCs
		\begin{eqnarray}
			c(0,t)=0 \ \text{and} \ c(1,t)=1
		\end{eqnarray}
		and IC $c(x, 0)= c_0.$ Plot a few concentration profiles.
	\end{exercise}

	\begin{exercise}
		If you have not done so already, show that Eq. (\ref{eq:soldifdibc})
		actually is a solution to the diffusion equation
		with zero Dirichlet BCs. (Hint: Substitute the solution into the
		diffusion equation and compare the two sides of the equation.)
		Then give a necessary condition for the existence of the solution. 
	\end{exercise}

	\begin{exploration}
		\label{exercise:numerics}
		In the next chapter we deal with non-linear problems and sometimes we need to  
		perform \emph{supportive numerical explorations} of such systems.  
		Since we now have the analytical solution for the (linear) diffusion equation, the time is right 
		to implement a numerical scheme which can also handle general reaction-diffusion problems. 
		Once we have checked the implementation against known solutions, we can,  
		carefully, use the numerical method for non-linear problems as well. Go to Appendix \ref{sect:ftcs}.
	\end{exploration}

	\begin{exploration} \label{expl:diffneumann}
		(Mandatory) Consider the diffusion equation
		\begin{eqnarray}
			\frac{\partial c}{\partial t} = D\frac{\partial^2c}{\partial x^2} \, , 
		\end{eqnarray}
		with IC $c(x,0)=f(x)$ and Neumann BCs
		\begin{eqnarray}
			\left.\frac{\partial c}{\partial x}\right|_{x=0}=\left.\frac{\partial c}{\partial x}\right|_{x=L}=0 \, .
		\end{eqnarray}
		Use seperation of variable to show the solution to this is	given by
		\begin{eqnarray}
			c(x,t)=\frac{a_0}{2} + \sum_{n=1}^\infty a_n e^{-(n\pi/L)^2Dt}  
			\cos\left(\frac{n\pi}{L} x\right) \, ,
		\end{eqnarray}
		with coefficients
		\begin{eqnarray}
			a_n = \frac{2}{L}\int_0^Lf(x)\	\cos\left(\frac{n\pi}{L} x\right) \, \d x \, .
		\end{eqnarray}
		for $n \in \mathbb{N}_+$.

		\textit{Optional}: Implement the Neumann BC into the numerical code 
		from Exploration \ref{exercise:numerics}. 
		Test your implementation against a known solution to a problem of your choice.
	\end{exploration}

\end{exerciseregion}

\section{Unbounded domains and Fourier transforms}
Up till now we have dealt with Dirichlet, Neumann, or mixed boundary conditions. 
These boundaries characterise bounded systems (or domains), where we can clearly define some properties at the ends of the tube. 
In certain cases we have that the tube length is large compared to (i) the initial concentration profile and 
(ii) the characteristic \emph{diffusion length scale}, $L_D(t) = \sqrt{4 D t}$. 
The second condition means that we limit our selves to study diffusion on sufficiently small times. 
We then apply \emph{unbounded domain} BCs    
\begin{eqnarray}
	\lim_{x \rightarrow \pm \infty} c(x,t) = 0  \, .
\end{eqnarray}
Notice that for unbounded domains we shift the coordinate systems such that we have $-\infty < x < \infty$ 
rather than $0 \leq x < \infty$. As for bounded BCs we need some prerequisites 
before we can solve the diffusion equation with unbounded BCs. 

First, we re-write the Fourier series in a complex form. Recall, in terms of wave vector 
$k_n$ the Fourier series is given by
\begin{eqnarray}
	F(x) = \frac{a_0}{2} + \sum_{n=1}^\infty a_n\cos(k_n x) + \sum_{n=1}^\infty b_n \sin(k_n x) \, .
\end{eqnarray}
We have the following relations
\begin{eqnarray}
	\label{eq:euler2ndidentity}
	\cos(k_n x) &=& \frac{e^{ik_n x} + e^{-ik_nx}}{2}  \nonumber \\
	\sin(k_n x) &=& \frac{e^{ik_n x} - e^{-ik_nx}}{2i}   
\end{eqnarray}
which can shown directly by Euler's identity. Substitution leads to 
\begin{eqnarray}
	F(x) &=&  \frac{a_0}{2}+ \sum_{n=1}^\infty\frac{a_n}{2}(e^{ik_nx}+e^{-ik_nx}) + \sum_{n=1}^\infty \frac{b_n}{2i} ( e^{ik_nx}-e^{-ik_nx}) 
	\nonumber \\
	 &=&  \frac{a_0}{2}+ \sum_{n=1}^\infty\frac{ia_n+b_n}{2i}\, e^{ik_nx} + \sum_{n=1}^\infty \frac{ia_n - b_n}{2i} \, e^{-ik_nx} 
	 \label{eq:FEuler} \, .
\end{eqnarray}
Note, we here assumed that we can interchange the terms in the series. The wave vector has the property that
\begin{eqnarray}
	k_{-n} = \frac{-n\pi}{L} = -k_n
\end{eqnarray}
and therefore we can change the index of the last term in Eq. (\ref{eq:FEuler})
\begin{eqnarray}
	\sum_{n=1}^\infty \frac{ia_n - b_n}{2i} \, e^{-ik_nx} =
	\sum_{n=-\infty}^{-1} c_n \, e^{ik_nx}.  
\end{eqnarray}
Inserting into Eq.
(\ref{eq:FEuler})  
\begin{eqnarray}
	F(x) = \sum_{-\infty}^\infty c_n e^{i k_n x}
\end{eqnarray}
where $c_n \in \mathbb{C}$ are the \emph{complex Fourier coefficients} which 
are related to the real Fourier coefficients by
\begin{eqnarray}
	c_n = \left\{
		\begin{array}{ll}
			\frac{ia_{|n|} - b_{|n|}}{2i} \ & \text{if} \ n < 0 \\
			\frac{ia_{n} + b_{n}}{2i} \ & \text{if} \ n > 0 \\
			\frac{a_0}{2} \ & \text{if} \ n = 0
		\end{array}	
	\right.
\end{eqnarray}
The complex Fourier coefficients are found from the orthogonal property of the exponential function on the
usual interval $[-L;L]$.
\begin{theorem}
	If the complex Fourier series representation of $f:\mathbb{R} \rightarrow \mathbb{R}$ equals the function, i.e., if 
	\begin{eqnarray}
		f(x) = \sum_{-\infty}^\infty c_n e^{i k_n x}
	\end{eqnarray}
	then
	\begin{eqnarray}
		\langle f(x), e^{-ik_n x} \rangle = 2 L c_n  
	\end{eqnarray}
	on $I=[-L; L]$. Moreover, $\langle f(x), e^{-ik_m x} \rangle = 0$ on $I$ if $m \neq n$. 
\end{theorem}

\begin{proof}
	See exercise \label{missingproof}
\end{proof}

In unbounded domains we cannot general resolve the function in periods of finite $L$. The (complex) Fourier
coefficients are then defined to be the limiting value of
\begin{eqnarray}
	\label{eq:cnunbound}
	c_n = \lim_{L\rightarrow \infty} \frac{1}{2L} \int_{-L}^L f(x) e^{-ik_n x}
	\, \d x  
\end{eqnarray}
if the integral exists. This definition is not as innocent as it may first appear since the wave vector is dependent on 
both the index $n$ and $L$, and the limit is not always easy to evaluate even if the integral can be found. 
However, Eq. (\ref{eq:cnunbound}) leads to another very powerful definition, namely, the \emph{Fourier transform}
\begin{eqnarray}
	\fourier{f}(k_n) = \int_{-\infty}^\infty  f(x) e^{-ik_n x} \, \d x \, .
\end{eqnarray}
The Fourier transform is an \emph{integral transform} or \emph{integral operator}, that transforms the function $f$ into 
another function $\fourier{f}$ that depends on the wave vector $k_n$. This transform is defined for any (real) 
wave vector, and often the subscript $n$ is omitted. 

\begin{example}
	Find the Fourier transform $\fourier{f}$ for 
	\begin{eqnarray}
		f(x) = \left\{
			\begin{array}{ll}
				e^{-a x} & \ \text{if} \ x \geq 0 \\
				0 & \ \text{if} \ x < 0
			\end{array}
		\right.
	\end{eqnarray}
	where $a > 0$. From the definition
	\begin{eqnarray}
		\fourier{f}(k) &=& \int_{-\infty}^\infty e^{-ax}  e^{-ik x} \, \d x = \int_0^\infty e^{-(a+ik)x}\, \d x 
		\nonumber \\
		&=& - \frac{1}{a+ik} \left[e^{-ax}(\cos(kx) - i\sin(kx))\right]_{0}^\infty = \frac{1}{a+ik} \, .
	\end{eqnarray}
	Notice $\fourier{f}$ is a complex valued function. 
\end{example}	

\noindent The Fourier transform is often written in terms of the operator symbol $\mathcal{F}$, that is, 
\begin{eqnarray}
	\fourier{f}(k) = \mathcal{F}[f](k) \, .
\end{eqnarray}
The Fourier transform has some very powerful properties which we will exploit.  

\begin{theorem}
	\begin{description}[style=unboxed,leftmargin=0.4cm]
		\item[1] {
			If $f,g:\mathbb{R} \rightarrow \mathbb{R}$ are bounded and piece wise continuous, then 
			\begin{equation}
				\mathcal{F}[\alpha f + g] = \alpha \mathcal{F}[f] + \mathcal{F}[g] 
			\end{equation}
			for  $\alpha \in \mathbb{R}.$ This is the linearity property. 
		}
		\item[2] {
		If $f: \mathbb{R} \rightarrow \mathbb{R}$ is at least twice differentiable, bounded and 
		$\lim_{x \rightarrow \pm \infty} f = 0$, then   
			\begin{equation}
			\mathcal{F}\left[\frac{\d f}{\d x}\right] = ik \mathcal{F}[f] \ \ \text{and} 
			\ \ \mathcal{F}\left[\frac{\d^2 f}{\d x^2}\right] = -k^2 \mathcal{F}[f]
			\end{equation}
		}
		\item[3]{ 
		Let $f=f(x,t)$ be defined for all positive $t$ and $x\in \R$. Assume $f$ has partial derivative with respect 
			to $t$, that $f$ is continous with respect to $x$ and $t$, is uniform convergent, and bounded. Then  
		\begin{equation}
			\label{eq:ftransprop3}
			\mathcal{F}\left[\frac{\partial f}{\partial t}\right] = \frac{\partial}{\partial t} \mathcal{F}[f].
		\end{equation}
			}
	\end{description}	
\end{theorem}

\begin{proof}
	The linearity follows directly from the linearity for integrals. The second property can be shown by integration by parts 
	\begin{eqnarray}
		\mathcal{F}\left[\frac{\d f}{\d x}\right] &=& \int_{-\infty}^\infty f'(x) e^{-ikx} \, \d x  \nonumber \\
		&=& \left[f(x) e^{-ikx}\right]_{-\infty}^\infty + ik \int_{-\infty}^\infty f(x) e^{ikx}\, \d x = ik \fourier{f}(k)
	\end{eqnarray}
	since $\lim_{x\rightarrow \pm \infty} f = 0$. For the second order derivative integration by part is applied twice.

	For the third property we define the function $w(x,t) = f(x,t)e^{-ikx}$ and show that the right-hand side in Eq. (\ref{eq:ftransprop3}) 
	equals the left-hand side. Using the definition of $w$ and the partial derivative
	\begin{eqnarray}
		\frac{\partial}{\partial t} \mathcal{F}[f] &=& \frac{\partial}{\partial t} \int_{-\infty}^\infty w(x,t) \, \d x \nonumber  \\ 
		&=& \lim_{h \rightarrow 0} \frac{1}{h}\left\{ \int_{-\infty}^\infty w(x, t+h) \, \d x - \int_{-\infty}^\infty w(x, t) \, \d x\right\} 
		\nonumber \\
		&=&  \lim_{h \rightarrow 0} \int_{-\infty}^\infty \frac{w(x,t+h)-w(x,t)}{h}\, \d x 
	\end{eqnarray}
	From the property of $f$ we can interchange the limit and intergral
	\begin{eqnarray}
		\int_{-\infty}^\infty \lim_{h \rightarrow 0} \frac{w(x,t+h)-w(x,t)}{h}\, \d x 
		= \int_{-\infty}^\infty \frac{\partial w}{\partial t} \, \d x = \int_{-\infty}^\infty \frac{\partial f}{\partial t} e^{-ik x} \, \d x 
	\end{eqnarray}
	which is the left-hand side of Eq. (\ref{eq:ftransprop3}).	
\end{proof}

\noindent We can use the properties of the Fourier transform to the diffusion equation with unbounded BCs. 
Letting the operator act on both sides
\begin{eqnarray}
	\mathcal{F}\left[\frac{\partial c}{\partial t}\right] = \mathcal{F}\left[D \frac{\partial^2c}{\partial x^2}\right]
\end{eqnarray}
implies that 
\begin{eqnarray}
	\frac{\partial}{\partial t}  \fourier{c}(k, t) = -D k^2 \fourier{c} \, ,
\end{eqnarray}
for all $k$. In this way, the we transform the partial differential equation for $c$ 
into a differential equation for $\fourier{c}$, but with respect to a single variable. The general solution we know 
\begin{eqnarray}
	\label{eq:fouriersol}
	\fourier{c}(k, t) = \fourier{c}_0 e^{-Dk^2t} \, .
\end{eqnarray}
This is referred to \emph{the solution in Fourier space} \footnote{We have not formally defined this "space", 
but we will use the term nevertheless.}. 

If we wish a particular solution we of course need to specify the initial condition. For example, if we have an initial concentration 
which is Gaussian distributed with respect to $x$ 
\begin{eqnarray}
	\label{eq:initRealUnbounded}
	c(x, 0) = c_0 e^{-(x/\sigma)^2} \ , 
\end{eqnarray}
where $\sigma \in \R$ is a measure of the Gaussian width and $c_0$ the concentration at $x=0$, then 
\begin{eqnarray}
	\fourier{c}_0 = c_0 \int_{-\infty}^\infty e^{-(x/\sigma)^2 - ikx} \, \d x \, .
\end{eqnarray}
The trick is now to rewrite the exponent into another form
\begin{eqnarray}
	-\frac{x^2}{\sigma^2} - ikx  = -\frac{1}{\sigma^2} \left(x^2 + ik\sigma^2x \right)=  -\frac{1}{\sigma^2} 
	\left[ \left(x + \frac{ik\sigma^2}{2} \right)^2 + \frac{k^2\sigma^4}{4} \right] \, .
\end{eqnarray}
Substitution and the Fourier transform now reads
\begin{eqnarray}
	\fourier{c}_0 = c_0 e^{-\frac{k^2\sigma^2}{4}} \int_{-\infty}^\infty e^{-\frac{1}{\sigma^2}\left(x + \frac{ik\sigma^2}{2} \right)^2} \d x = 
	c_0 \sqrt{\pi} \sigma  e^{\frac{k^2\sigma^2}{4}} 
\end{eqnarray}
applying the useful Gaussian integral formula
\begin{eqnarray}
	\int_{-\infty}^\infty e^{-a(x+b)^2} \, \d x = \sqrt{\frac{\pi}{a}} \, .
\end{eqnarray}
Substitution into Eq. (\ref{eq:fouriersol}) we get 
\begin{eqnarray}
	\label{eq:fouriersolc}
	\fourier{c}(k,t)=c_0 \sqrt{\pi} \sigma  e^{\frac{-k^2\sigma^2}{4}}   e^{-Dk^2t}
\end{eqnarray}
Some times the solution in Fourier space can give us the information we seek, however, we often also want 
the solution to $c$ and not only $\fourier{c}$. We now define the \emph{inverse Fourier transform} of $\fourier{f}$ 
to be 
\begin{eqnarray}
	f(x) = \frac{1}{2\pi} \int_{-\infty}^\infty \fourier{f}(k) e^{ikx} \, \d k \, .
\end{eqnarray}
In operator notation we write this as $f(x) = \mathcal{F}^{-1}[\fourier{f}](x)$. 
Let us see a useful example. 
\begin{example}
	Let $f(x) = e^{-a x^2}$, where $a>0$. Rewritting the exponent and using Gaussian's integration formula
	\begin{equation}
		\label{eq:expx2F}
		\fourier{f}(k)= \sqrt{\frac{\pi}{a}} e^{-k^2/4 a} \, .
	\end{equation}
	The inverse Fourier transform of $\fourier{f}$ is
	\begin{eqnarray}
		\frac{1}{2\sqrt{a \pi}} \int_{-\infty}^\infty e^{-k^2/4a + ikx} \, \d k &=&
			\frac{e^{-ax^2}}{2\sqrt{a \pi}} \int_{-\infty}^\infty e^{-\frac{1}{4a}\left(k + 2aix\right)^2} \, \d k \nonumber \\
			&=& \frac{\sqrt{4 a \pi}}{2\sqrt{a\pi}} e^{-ax^2} = e^{-ax^2} 
	\end{eqnarray}
	using the identity 
	\begin{equation}
		-k^2/4a + ikx = -\frac{1}{4a}\left(k^2 - i4akx\right) =   -\frac{1}{4a}\left(k + i2ax\right)^2 - ax^2 \, .
	\end{equation}
	Notice that we recover the original function $f(x) = e^{-ax^2}$. 
\end{example}	

\begin{wrapfigure}{l}{0.4\textwidth}
	\centering
	\includegraphics[width=0.35\textwidth]{figs/diffunbounded.eps}
	%\caption{\label{fig:img2} Perturbation}
\end{wrapfigure}
\paragraph{}
\vspace*{-\parskip}
The example is not a coincidence, of course, and is generalized in the following statement (which we will not prove).
If $f: \R \rightarrow \R$ is piecewise continous and bounded, then
\begin{eqnarray}
	\mathcal{F}^{-1}[\mathcal{F}[f]](x) = \mathcal{F}^{-1}[\fourier{f}](x) = f(x) \, .
\end{eqnarray}
This is the important \emph{inverse Fourier theorem}. 

Equation (\ref{eq:fouriersolc}) is continous and bounded thus the inverse Fourier theorem qaranties that we recover $c$ by 
an inverse transform. We can rewrite Eq. (\ref{eq:fouriersolc}) to 
\begin{eqnarray}
	\fourier{c}(k, t) = c_0 \sqrt{\pi} \sigma e^{-\alpha k^2} ,
\end{eqnarray}
with $\alpha = \frac{\sigma^2}{4} + Dt$. This is on the same form as Eq. (\ref{eq:expx2F}), and we can simply substitue and get
\begin{eqnarray}
	\label{eq:solcfinal}
	c(x,t)=\frac{c_0 \sigma}{\sqrt{\sigma^2 + 4 D t}} \, e^{-x^2/(\sigma^2 + 4Dt)} \, .
\end{eqnarray}
This is the solution\footnote{finally..!} to the diffusion equation in an unbounded domain with initial conditions given by 
Eq. (\ref{eq:initRealUnbounded}).

\begin{exerciseregion}

	\begin{exercise}
		Confirm Eq. (\ref{eq:solcfinal}).
	\end{exercise}
	
	\begin{exercise}
		Prove the theorem on page \pageref{missingproof}. 
	\end{exercise}

	\begin{exercise}
		Find the Fourier transform for 
		\begin{eqnarray}
			f(x) =
			\left\{
		    \begin{array}{ll}
		        -x^2+1 & -1 \leq x \leq 1 \\
					0 &  \text{otherwise}
			\end{array}
			\right.
		\end{eqnarray}
		and plot the function.
	\end{exercise}

	\begin{exploration}
		Recall the diffusion equation in polar coordinates and with angular symmetry 
		\begin{equation}
			\label{eq:rdpol_2}
				\frac{\partial c}{\partial t} = 
			\frac{D}{r}\frac{\partial}{\partial r}
			\left(  r\frac{\partial c}{\partial r} \right) \, .
		\end{equation}
		Envision an experiment where we add a drop of a chemical substance into the 
		center of a Petri dish that is filled with a solvent. 
		We assume that the radius of the Petri dish is large 
		compared to initial substance concentration distribution and that the diffusion length scale $L_D$ 
		is small. 

		When can we use Eq. (\ref{eq:solcfinal}) as an approximation to the solution 
		to this diffusion system? Solve Eq. (\ref{eq:rdpol_2}) numerically using the 
		FCTS alogrithm and compare with your answer.
	\end{exploration}
\end{exerciseregion}

