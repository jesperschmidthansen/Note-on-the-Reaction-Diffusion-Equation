
\begin{center}{\Large{\textbf{Preface to note}}}\end{center}

\bigskip 

\noindent One purpose of mathematical modelling is to obtain a deeper understanding of 
a system or phenomenon. 
The new insight will, hopefully, enable more precise system predictions and control. 
Mathematical modelling can also motivate one to acquire new mathematical skills, and this gives rise 
to explorations of fundamental mathematics. 

This text seeks to give an introduction to mathematical modelling of reaction-diffusion systems. The focus is on 
phenomenology and not on providing detailed quantitative models of specific problems. In the exploration we will 
need and therefore require skills of 
\begin{itemize}
	\item boundary value problems,
	\item the regular perturbation method,
	\item Fourier series and the Fourier transform,
	\item the diffusion equation (example of a parabolic differential equation),
	\item and some particular techniques to analyse the reaction-diffusion equation.
\end{itemize}
The last point includes simple numerical analysis: the numerical algorithms are 
found in the appendices and code can downloaded from the course moodle webpage.  

There are, of course, many different ways we can approach our exploration, but 
there are some obvious choices based on \emph{the principle of progression}. For example, we can 
use some important results from pure diffusion problems to solve the linear reaction-diffusion equation, hence, it 
will be meaningful to treat the former first. On this basis, the text is based on three 
main topics, namely,  
\begin{center}
	the steady state $\rightarrow$ the diffusion equation $\rightarrow$ the
	react. diff. equation.	
	\begin{tikzpicture}
		\draw(1,0.5) -- (1,0);
		\draw(1,0) -- (9,0.0);
		\draw[->](9,0.0) -- (9,0.5);
	\end{tikzpicture}
\end{center}
The arrows indicate depedendencies. 

Our treatment is based on examples, questions, exercises, and, to some extend, proofs of theorems. It is not 
the purpose here to give a formal theoretical foundation of the different topics; this calls for a much more 
rigourus approach and require advanced knowledge of, for example, series convergence and functional analysis.  
The theorems will therefore often only require so-called mild conditions and the proofs will be of the "calculus"-type. 
In this way, the note follows the standard format for many popular texts on mathematical biology.  
I will refer the interest reader to the more advanced literature whenever I can. 

The prerequisite needed to read the note is knowledge of (i) ordinary differential 
equations and initial value problems, (ii) dynamical systems analysis (e.g. phase portrait analysis), 
and (iii) calculus and basic linear algebra. 


